\documentclass[a4paper]{article}

\usepackage{amsfonts, amsmath, amssymb, amsthm}
\usepackage{graphicx}
\usepackage{fullpage}
\usepackage{float}

\usepackage{tikz}
\usetikzlibrary{calc}

\newtheorem{lemma}{Lemma}
\newtheorem{theorem}{Theorem}
\newtheorem{corollary}{Corollary}
\newtheorem{definition}{Definition}
\newtheorem{conjecture}{Conjecture}

\newcommand{\Z}{\mathbb{Z}}
\newcommand{\N}{\mathbb{N}}
\renewcommand{\qedsymbol}{$\blacksquare$}
\newcommand{\abs}[1]{\left| #1 \right|}
\renewcommand{\dim}[1]{\textnormal{dim}\left( #1 \right)}

\setlength{\parindent}{0em}
\setlength{\parskip}{1em}

\begin{document}
	\title{Conjectures Related to \textit{Lights Out}}
	\author{William Boyles}
	\date{\today}
	\maketitle
	
	\section{Definitions}
	Let \textit{Lights Out} be a game played on a simple graph $G = (V,E)$ where clicking a vertex $v$ toggles the on/off state of $v$ and its neighbors.
	One wins the game by finding a sequence of clicks that turns off all the lights.
	
	\begin{definition}
		The Most Clicks Problem (MCP) asks, for playing on some graph $G$, what is the most clicks needed to solve any solvable starting configuration, assuming that one solves each configuration in as few clicks as possible.
	\end{definition}
	That is, what is the most number of moves one needs to be able to solve any solvable initial configuration?
	This question is akin to looking for ``God's Number'' on a Rubik's Cube.

	\begin{definition}
		Let $d(n) = \dim{\ker{(A + I)}}$ over the field $\Z_2$ where $A$ is the adjacency matrix of $G$ an $n \times n$ grid.
	\end{definition}
	You may also hear $d(n)$ referred to as the ``nullity'' of the $n \times n$ grid.
	The right-hand side of the equation applies for any graph, not just grids.
	Elements of the null space correspond to sets of clicks that have no net effect on the states of the vertices when applied.
	These elements also correspond exactly to subsets of $V$ such that all vertices have an even number of elements from the subset in their neighborhood (themselves and their neighbors).
	Thus, these elements of the null space may be referred to as ``null patterns'' or ``quiet patterns''.
	
	\begin{definition}\label{def-polys}
		Let $f_n(x)$ be the degree $n$ polynomial in the ring $\Z_2[x]$ such that
		\begin{equation*}
			f_n(x) = \begin{cases}
				1 & n = 0 \\
				x & n = 1 \\
				x f_{n-1}(x) + f_{n-2}(x) & n > 1
			\end{cases}.
		\end{equation*}
	\end{definition}
	These polynomials become useful in calculating $d(n)$.
	To find $f_n(x)$, you can truncate elementary cellular automaton rule 90 after $2^k > n$ rows and look at the $n$-th column from the right (rightmost column is 0) from bottom to top.
	This will give you the coefficients of $f_n(x)$ (1 for black, 0 for white) starting from $x^0$ and going up to $x^n$.

	\section{Existing Results}
	\begin{theorem}[Sutner]
		For an $n \times n$ grid, the answer to the MCP is $n^2$ if and only if $d(n) = 0$.
	\end{theorem}

	\begin{theorem}[Boyles]\label{thm-polys-div-x}
		The polynomial $f_n(x+1)$ is divisible by $x$ if and only if $n \equiv 2 \mod 3$. 
	\end{theorem}
	\begin{proof}
		By induction.
		For the base case, notice the following are true:
		\begin{align*}
			f_0(x+1) &= 1 \\
			f_1(x+1) &= x+1 \\
			f_2(x+1) &= x^2.
		\end{align*}
	
		From Definition \ref{def-polys}, we see that
		\begin{equation*}
			f_{n}(x+1) = (x+1)f_{n-1}(x+1) + f_{n-2}(x+1).
		\end{equation*}
	
		Assume that this theorem holds true for all indices up to and including $n-1$.
		Now for the inductive step there are 3 cases to consider.
		
		\textbf{Case 1: } $n \equiv 2 \mod 3$. \\
		So, $n-1$ and $n-2$ are not equivalent to 2 modulo 3 and therefore contain a 1's term.
		Thus, we can write
		\begin{align*}
			f_{n-1}(x+1) &= 1 + x p(x) \\
			f_{n-2}(x+1) &= 1 + x q(x),
		\end{align*}
		Thus,
		\begin{align*}
			f_{n}(x+1) &= (x + 1)(1 + x p(x)) + 1 + x q(x) \\
				&= x^2 p(x) + x p(x) + x q(x) + x \\
				&= x(x p(x) + p(x) + q(x) + 1).
		\end{align*}
		We see that $f_n(x+1)$ is divisible by $x$, as desired.
		
		\textbf{Case 2: } $n \equiv 1 \mod 3$. \\
		In this case we have $n-2$ equivalent to 2 modulo 3 while $n-1$ is not.
		Therefore, $f_{n-2}(x+1)$ is divisible by $x$, while $f_{n-1}(x+1)$ is not and therefore must contain a 1's term.
		So, we can write
		\begin{align*}
			f_{n-1}(x+1) &= 1 + x p(x) \\
			f_{n-2}(x+1) &= x q(x).
		\end{align*}
		Thus,
		\begin{align*}
			f_n(x+1) &= (x+1)(1 + xp(x)) + xq(x) \\
				&= x^2 p(x) + xp(x) + xq(x) + x + 1.
		\end{align*}
		We see that $f_n(x+1)$ contains a 1's term and is therefore not divisible by $x$, as desired.
		
		\textbf{Case 3: } $n \equiv 0 \mod 3$. \\
		In this chase we have $n-1$ equivalent to 2 modulo 3 while $n-2$ is not.
		Therefore, $f_{n-1}(x+1)$ is divisible by $x$, while $f_{n-2}(x+1)$ is not and therefore must contain a 1's term.
		So, we can write
		\begin{align*}
			f_{n-1}(x+1) &= x p(x) \\
			f_{n-2}(x+1) &= 1 + x q(x).
		\end{align*}
		Thus,
		\begin{align*}
			f_n(x+1) &= (x+1)(xp(x)) + 1 + xq(x) \\
			&= x^2 p(x) + xp(x) + xq(x) + 1.
		\end{align*}
		We see that $f_n(x+1)$ contains a 1's term and is therefore not divisible by $x$, as desired.
	\end{proof}
	
	\begin{theorem}[Hunziker, Machivelo, Park]\label{thm-polygcd}
		For all $n \in \N$,
		\begin{equation*}
			d(n) = \deg{\gcd{\left(f_n(x), f_n(x+1)\right)}}.
		\end{equation*}
	\end{theorem}
	This theorem gives an another way to calculate $d(n)$.
	
	\begin{theorem}[Sutner]\label{thm-pow(2,k)-1}
		For all $k \in \N$,
		\begin{equation*}
			d(2^k - 1) = 0.
		\end{equation*}
	\end{theorem}
	This theorem shows that there are infinitely many $n$ such that $d(n) = 2$.
	
	\begin{theorem}[Sutner, Boyles]\label{thm-tiling}
		For all $n, k \in \N$,
		\begin{equation*}
			d(nk - 1) \geq d(n-1).
		\end{equation*}
	\end{theorem}
	This theorem allows us to find a lower bound on the nullity of larger boards by looking at the nullity of smaller boards.
	Intuitively, it takes advantage of the face that the geometry of a square grid allows us to tile small quiet patterns to make bigger ones.
	
	\begin{theorem}[Sutner, Boyles]
		For all $n \in \N$,
		\begin{align*}
			d(2n+1) &= 2d(n) + \delta_n \\
			\delta_n &\in \{0,2\} \\
			\delta_{2n+1} &= \delta_{n}.
		\end{align*}
		In particular,
		\begin{equation*}
			\delta_n = 2 \deg{\gcd{\left(x, \frac{f_n(x+1)}{\gcd{(f_n(x+1), f_n(x))}}\right)}}.
		\end{equation*}
	\end{theorem}
	Unlike Theorem \ref{thm-tiling}, this gives an exact relationship between the nullities of different-sized grids.
	Notice that once we calculate $d(1) = d(3) = 0$, Theorem \ref{thm-pow(2,k)-1} follows as a corollary.
	
	\section{Conjectures}
	
	\subsection{Formulas for $d(n)$}
	We'd like to extend Theorem \ref{thm-pow(2,k)-1} to other exponentials.
	When investigating for such patterns, it seemed to make the most sense to group these by prime factorization.
	Although we list many below, we suspect that are many more similar relations with prime factorization to be found.
	\begin{conjecture}\label{conj-many-formulas}
		For all $n,m,o \in \N$, the following hold:
		\begin{align*}
			d(2^n - 1) = 0 \\
			d(3^n - 1) &= 0 \\
			d(5^n - 1) &= 4 \\
			d(2^n * 3^m - 1) &= 2^{n+1} - 2 \\
			d(7^n - 1) &= 0 \\
			d(2^n * 5^m - 1) &= 2^{n+2} \\
			d(11^n - 1) &= 0 \\
			d(13^n - 1) &= 0 \\
			d(2^n * 7^m - 1) &= 0 \\
			d(3^n * 5^m - 1) &= 4 \\
			d(17^n - 1) &= 8 \\
			d(19^n - 1) &= 0 \\
			d(3^n * 7^m - 1) &= 24 \text{ for } n > 1, m > 0 \text{, else } 0 \\
			d(2^n * 11^m - 1) &= 0 \\
			d(23^n - 1) &= 0 \\
			d(2^n * 13^m - 1) &= 0 \\
			d(29^m - 1) &= 0 \\
			d(2^n * 3^m * 5^o - 1) &= 3*2^{m+1} - 2 \\
			d(31^n - 1) &= 20 \\
			d(3^n * 11^m - 1) &= 20 \\
			d(2^n * 17^m - 1) &= 2^{n+3} \\
			d(5^n * 7^m - 1) &= 4 \\
			d(37^n - 1) &= 0 \\
			d(2^n * 19^m - 1) &= 0 \\
			d(3^n * 13^m - 1) &= 0 \\
			d(41^n - 1) = 0 \\
			d(2^n * 3^m * 7^o - 1) &= 13*2^{n+1} - 2 \text{ for } m > 1, o > 0\text{, else } 2^{n+1} - 2  \\
			d(43^n - 1) &= 0 \\
			d(2^n * 23^m - 1) &= 0.
		\end{align*}
	\end{conjecture}
	Notice that except for $d(2^n * 3^m * 5^o - 1)$ that the outputs of no other right-hand sides depend on any other exponents except that on the 2.
	Notice that except for $d(2^n * 3^m * 5^o - 1)$ and $d(2^n * 3^m * 7^o)$ that no other statements rely on exponents other than those on the 2 in the output or conditionally.
	This may be a quirk that arises only when there are 3 prime factors and nothing special about these particular cases.
	Notice that except for $d(3^n * 7^m)$ and $d(2^n * 3^m * 7^o - 1)$ that no other outputs are conditional on the values of the exponents for which formula to use.
	This might just be a fluke with the cases listed, but we suspect that this has something to do with the number $21 = 3 * 7$.

	We can take certain parts of Conjecture \ref{conj-many-formulas} and state them more generally.
	\begin{conjecture}\label{conj-odds}
		Let $a$ be an odd number that is not divisible by 21.
		Then for all $n \in \N$,
		\begin{equation*}
			d(a^k - 1) = d(a-1).
		\end{equation*}
	\end{conjecture}
	We suspect that this conjecture is likely false, or at least, not completely descriptive of what's going on.
	A similar version of this conjecture with $k \geq k_0$ and $d(a^k - 1) = d(a^{k_0} - 1)$ seems to hold for all odd numbers, and it seems we only need $k_0 = 2$.
	We're not at all sure what's so special about the number 21.
	We're also not entirely sure that $a$ being odd is what truly matters, especially considering 21 is such an unexplained exception.
	Instead, we might be able to extend what $a$'s are applicable and deal with the exception of 21 by seeing when $d$ outputs 0 for exponents of 1 and when $\delta_{n}$ is 0 or 2.
	
	\subsection{Other Conjectures}
	Theorem \ref{thm-pow(2,k)-1} shows that there are infinitely many $n$ such that $d(n) = 0$.
	We'd like similar results for other even values, especially 2 since we've done some significant work in solving the MMP for nullity 2 boards.
	\begin{conjecture}\label{conj-inf-nullity-2}
		There are infinitely many $n \in \N$ such that $d(n) = 2$.
		In particular, for all $k \in \N$,
		\begin{equation*}
			d(2 * 3^k - 1) = 2.
		\end{equation*}
	\end{conjecture}
	Assuming that Conjecture \ref{conj-many-formulas} or \ref{conj-odds} is true for the case of $d(3^n - 1)$, $\delta_{3^n - 1}$ is 2 if and only $x$ divides $f_{3^n - 1}(x+1)$.
	By Theorem \ref{thm-polys-div-x} this is true.
	This this conjecture is implied by the $d(3^n - 1)$ case of Conjectures \ref{conj-many-formulas} and \ref{conj-odds}.
	
	Similar versions of Conjecture \ref{conj-inf-nullity-2} either following directly from parts of Conjectures \ref{conj-many-formulas} and \ref{conj-odds} or could be implied in a similar way to what we showed for nullity 2.
	Thus, the following conjecture may seem surprising.
	\begin{conjecture}
		There exists some even $k \in \N$ such that for all $n \in \N$,
		\begin{equation*}
			d(n) \neq k.
		\end{equation*}
		In particular, the smallest such $k$ is 26.
	\end{conjecture}
	That is, there are no nullity 26 grids.
	If we could show that all nullity 12 grids have $\delta = 0$, we'd show there are no odd grids with nullity 26.
	Other such $k$ we have found are 34, 52, 54, 68, and 70.
	
	Previously, we were able to solve the MCP for nullity 2 grids of size $6k-1$.
	We suspect that all nullity 2 grids are of this size, meaning we solved the MCP for all nullity 2 grids.
	\begin{conjecture}
		If $d(n) = 2$ then $n \equiv -1 \mod 6$.
	\end{conjecture}

	Our method for solving the MCP for nullity 2 grids of size $6k-1$ involved getting an exact solution to a integer linear program by noticing all constraints could be tightly satisfied at once.
	This allowed us to give a parabola in terms of $k$ that solved the MCP.
	However, this does not seem to work for higher nullities.
	We suspect that the answer to the MCP is still a parabola though.
	Thus, we can calculate the answer the the MCP for 3 boards of the same nullity in the same family (i.e. size is same mod some number) and derive the parabola.
	\begin{conjecture}
		Regarding the answer to the MCP for for an $n \times n$ grid,
		\begin{itemize}
			\item
				If $n = 6k-1$ and $d(n) = 2$, then the answer to the MCP is $26k^2 - 12k + 1$.
			\item
				All nullity 2 grids are of size $6k-1$.
			\item
				If $n = 10k - 6$ and $d(n) = 4$, then the answer to the MCP is $17k^2 - 10k$.
			\item
				All nullity 4 grids are of size $10k - 6$
			\item
				If $n = 24k - 13$ and $d(n) = 6$, then the answer to the MCP is $88k^2 - 24k + 1$.
			\item
				All nullity 6 grids are of size $24k-13$.
			\item
			 	If $n = 40k - 31$ and $d(n) = 8$, then the answer to the MCP is $60k^2 - 20k - 3$.
			 \item
			 	If $n = 34k - 18$ and $d(n) = 8$, then the answer to the MCP is $161k^2 - 34k - 3$.
			 \item
			 	All nullity 8 grids are either of side $40k-31$ or $34k-18$.
			 \item
			 	If $n = 60k - 31$ and $d(n) = 10$ then the answer to the MCP is $506k^2 - 60k - 3$.
			 \item
			 	All nullity 10 grids are of size $60k - 31$.
			 \item
			 	If $n = 340k - 256$ and $d(n) = 12$, then the answer to the MCP is $3761k^2 - 169k - 13$.
			 \item
			 	All nullity 12 grids are of size $340k - 256$.
		\end{itemize}
	\end{conjecture}
	Notice that there are different ``families'' associated with each nullity.
	These families correspond to a certain size of board.
	Notice that one nullity might have multiple families.
	It seems that the smallest board in each family appears in OEIS sequence A118142.
	We suspect that the parabolas giving the answer to the MCP for a given family always have integer coefficients.
\end{document}
