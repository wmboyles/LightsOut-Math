\documentclass[a4paper]{article}

\usepackage{amsfonts, amsmath, amssymb, amsthm}
\usepackage{graphicx}
\usepackage{fullpage}
\usepackage{float}

\newtheorem{lemma}{Lemma}
\newtheorem{theorem}{Theorem}
\newtheorem{corollary}{Corollary}
\newtheorem{conjecture}{Conjecture}
\newtheorem{example}{Example}

\newcommand{\Z}{\mathbb{Z}}
\newcommand{\N}{\mathbb{N}}
\renewcommand{\qedsymbol}{$\blacksquare$}
\newcommand{\abs}[1]{\left| #1 \right|}
\renewcommand{\dim}[1]{\text{dim}\left( #1 \right)}
\newcommand{\xor}{\oplus}

\setlength{\parindent}{0em}
\setlength{\parskip}{1em}

\begin{document}
	\title{Resolution to Sutner's Conjecture}
	\author{William Boyles\thanks{Department of Mathematics, North Carolina State University, Raleigh, NC 27695 (wmboyle2@ncsu.edu)}}
	\date{\today}
	\maketitle
	
	\section{Introduction}
	Consider a game played on a simple graph $G = (V,E)$ where each vertex consists of a clickable light.
	Clicking any vertex $v$ toggles on on/off state of $v$ and its neighbors.
	One wins the game by finding a sequence of clicks that turns off all the lights.
	When $G$ is a $5 \times 5$ grid, this game was commercially available from Tiger Electronic as \textit{Lights Out}.
	
	Sutner was one of the first to study these games mathematically.
	He showed that for any $G$ the initial configuration of all lights on is solvable \cite{Sutner1989}.
	He also found that when $d(G) = \dim{\ker{(A + I)}}$ over the field $GF(2)$, where $A$ is the adjacency matrix of $G$, is 0 all initial configurations are solvable.
	In particular, 1 out of every $2^{d(G)}$ initial configurations are solvable, while each solvable configuration has $2^{d(G)}$ distinct solutions \cite{Sutner1989}.
	When investigating $n \times n$ grid graphs, Sutner conjectured the following relationship:
	\begin{align*}
		d_{2n+1} &= 2d_n + \delta_n \text{, } \delta_n \in \{0,2\} \\
		\delta_{2n+1} &= \delta_n,
	\end{align*}
	where $d_n = d(G)$ for $G$ an $n \times n$ grid graph \cite{Sutner1989}.
	
	We resolve this conjecture in the affirmative.
	We use results from Sutner that give the nullity of a $n \times n$ board as the GCD of two polynomials in the ring $\Z_2[x]$ \cite{Sutner96sigma-automataand}.
	We then apply identities from Hunziker, Machiavelo, and Park that relate the polynomials $(2n+1) \times (2n+1)$ grids and $n \times n$ grids \cite{HUNZIKER2004465}.
	Finally, we use a result from Ore about the GCD of two products \cite{ore_number_theory}.
	Together, these results allow us to prove Sutner's conjecture and describe exactly when $\delta_n$ is 0 or 2.
	
	\section{Preliminary Results}
	Sutner showed how to calculate $d_n$ as the degree of the GCD of two polynomials in $\Z_2[x]$ \cite{Sutner96sigma-automataand}.
	
	\begin{theorem}[Sutner]\label{Sutner_gcd}
		Let $f_n(x)$ be the degree $n$ polynomial in the ring $\Z_2[x]$ defined recursively by
		\begin{equation*}
			f_n(x) = \begin{cases}
				1 & n=0 \\
				x & n=1 \\
				xf_{n-1}(x) + f_{n-2}(x) & \text{otherwise }.
			\end{cases}
		\end{equation*}
		Then for all $n \in \N$.
		\begin{equation*}
			d_n = \deg{\gcd\left(f_{n}(x), f_{n}(x+1)\right)}.
		\end{equation*}
	\end{theorem}

	This recursive definition gives a brute force approach to calculate $f_n(x)$.
	However, Hunziker, Machiavelo, and Park show the following identity that can make calculating certain $f_n(x)$ easier \cite{HUNZIKER2004465}.
	
	\begin{theorem}[Hunziker, Machiavelo, and Park]\label{HMP_identity}
		Let $n = b\cdot2^{k-1} - 1$ where $b, k \in \N$.
		Then
		\begin{equation*}
			f_n(x) = x^{2^{k-1}-1} f_{b-1}^{2^{k-1}}(x).
		\end{equation*}
	\end{theorem}

	In particular, we will use this result to relate $f_{2n+1}(x)$ and $f_{4n+3}(x)$ to $f_n(x)$.
	\begin{corollary}\label{cor1}
		The following identities hold
		\begin{align*}
			f_{2n+1}(x) &= xf_{n}^2(x) \\
			f_{4n+3}(x) &= x^3f_{n}^4(x).
		\end{align*}
	\end{corollary}
	\begin{proof}
		Notice that $2n+1 = (n+1)2^{2-1} - 1$ and $4n+3 = (n+1)2^{3-1} - 1$.
		Thus, our desired identities follow from Theorem \ref{HMP_identity}.
	\end{proof}

	Now that we have a way to express $f_{2n+1}(x)$ and $f_{4n+3}(x)$ as a product of $f_{n}(x)$ and a power of $x$, we simply need a way to express the GCD of products so we can calculate $d_n$.
	This is where a number-theoretic result from Ore comes in handy \cite{ore_number_theory}.
	
	\begin{theorem}[Ore]\label{Ore_gcd}
		Let $a$, $b$, $c$, and $d$ be integers.
		Let $(a,b)$ denote $\gcd{(a,b)}$.
		Then
		\begin{equation*}
			(ab,cd) = (a,c)(b,d)\left(\frac{a}{(a,c)},\frac{d}{(b,d)}\right)\left(\frac{c}{(a,c)},\frac{b}{(b,d)}\right).
		\end{equation*}
	\end{theorem}

	Ore's result deals specifically with integers.
	However, because both the integers and $\Z_2[x]$ are Euclidean domains, the result will still hold.

	\section{Proof of Sutner's Conjecture}
	Finally, we are ready to prove Sutner's conjecture \cite{Sutner1989}. 
	\begin{theorem}
		For all $n \in \N$,
		\begin{equation*}
			d_{2n+1} = 2d_n + \delta_n,
		\end{equation*}
		where $\delta_n \in \{0,2\}$, and $\delta_{2n+1} = \delta_n$.
	\end{theorem}
	\begin{proof}
		Let $(a,b)$ denote $\gcd{(a,b)}$.
		Applying the results from Theorems \ref{Sutner_gcd}, \ref{HMP_identity}, and \ref{Ore_gcd},
		\begin{align*}
			d_{2n+1} &= \deg \left(f_{2n+1}(x), f_{2n+1}(x+1)\right) \\
				&= \deg \left(xf^2_n(x), (x+1)f^2_n(x+1)\right) \\
				&= \deg (x,x+1)\left(f^2_n(x),f^2_n(x+1)\right)\left(\frac{x+1}{(x,x+1)},\frac{f^2_n(x)}{(f^2_n(x),f^2_n(x+1))}\right)\left(\frac{x}{(x,x+1)},\frac{f^2_n(x+1)}{(f^2_n(x),f^2_n(x+1))}\right) \\
				&= \deg \left(f_n(x),f_n(x+1)\right)^2\left(x+1,\frac{f^2_n(x)}{(f_n(x),f_n(x+1))^2}\right)\left(x,\frac{f^2_n(x+1)}{(f_n(x),f_n(x+1))^2}\right) \\
				&= \deg \left(f_n(x),f_n(x+1)\right)^2\left(x+1,\frac{f_n(x)}{(f_n(x),f_n(x+1))}\right)\left(x,\frac{f_n(x+1)}{(f_n(x),f_n(x+1))}\right) \\
				&= 2d_n + \deg\left(x+1,\frac{f_n(x)}{(f_n(x),f_n(x+1))}\right)\left(x,\frac{f_n(x+1)}{(f_n(x),f_n(x+1))}\right).
		\end{align*}
		Notice that if we substitute $x+1$ for $x$,
		\begin{equation*}
			\left(x+1,\frac{f_n(x)}{(f_n(x+1),f_n(x)}\right) \text{ becomes } \left(x,\frac{f_n(x+1)}{(f_n(x),f_n(x+1))}\right).
		\end{equation*}
		Thus, we see that these two remaining GCD terms are either both 1 nor not 1 simultaneously.
		This means we can further simplify to
		\begin{equation*}
			d_{2n+1} = 2d_{n} + 2\deg\left(x,\frac{f_n(x+1)}{(f_n(x),f_n(x+1))}\right).
		\end{equation*}
		So, we see that
		\begin{equation*}
			d_{2n+1} = 2d_{n} + \delta_n \text{, where }\delta_n = 2\deg\left(x,\frac{f_n(x+1)}{(f_n(x),f_n(x+1))}\right).
		\end{equation*}
		Thus, $\delta_n \in \{0,2\}$.
		
		What remains is to show that $\delta_{n} = \delta_{2n+1}$.
		Applying Corollary \ref{cor1},
		\begin{align*}
			d_{4n+3} &= \deg\left(x^3f^4_n(x), (x+1)^3f^4_n(x+1)\right) \\
				&= \deg\left(x,(x+1)^3\right)\left(f^4_n(x),f^4_n(x+1)\right)\left(x^3,\frac{f^4_n(x+1)}{(f^4_n(x),f^4_n(x+1))}\right)\left((x+1)^3,\frac{f^4_n(x)}{(f^4_n(x),f^4_n(x+1))}\right) \\
				&= \deg\left(f_n(x),f_n(x+1)\right)^4\left(x^3,\frac{f^4_n(x+1)}{(f_n(x),f_n(x+1))^4}\right)\left((x+1)^3,\frac{f^4_n(x)}{(f_n(x),f_n(x+1))^4}\right) \\
				&= \deg\left(f_n(x),f_n(x+1)\right)^4\left(x^3,\frac{f^3_n(x+1)}{(f_n(x),f_n(x+1))^3}\right)\left((x+1)^3,\frac{f^3_n(x)}{(f_n(x),f_n(x+1))^3}\right) \\
				&= \deg\left(f_n(x),f_n(x+1)\right)^4\left(x,\frac{f_n(x+1)}{(f_n(x),f_n(x+1))}\right)^3\left((x+1),\frac{f_n(x)}{(f_n(x),f_n(x+1))}\right)^3 \\
				&= 4d_n + 3\delta_n.
		\end{align*}
		Also, from our work previously in this proof,
		\begin{align*}
			d_{4n+3} &= d_{2(2n+1) + 1} \\
				&= 2 d_{2n+1} + \delta_{2n+1} \\
				&= 2 \left(2d_{n} + \delta_{n}\right) + \delta_{2n+1} \\
				&= 4d_{n} + 2\delta_{n} + \delta_{2n+1}.
		\end{align*}
		For these two expressions for $d_{4n+3}$ to be equal, we must have $\delta_{2n+1} = \delta_n$, as desired.
	\end{proof}
	
	\newpage
	\bibliography{refs.bib}
	\bibliographystyle{amsplain}
\end{document}