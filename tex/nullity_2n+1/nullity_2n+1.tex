\documentclass[a4paper]{article}

\usepackage{amsfonts, amsmath, amssymb, amsthm}
\usepackage{graphicx}
\usepackage{fullpage}
\usepackage{float}
\usepackage{enumitem}

\newtheorem{lemma}{Lemma}
\newtheorem{theorem}{Theorem}[section]
\newtheorem{corollary}{Corollary}[theorem]
\newtheorem{conjecture}{Conjecture}
\newtheorem{example}{Example}
\newtheorem{remark}{Remark}

\newcommand{\Z}{\mathbb{Z}}
\newcommand{\N}{\mathbb{N}}
\renewcommand{\qedsymbol}{$\blacksquare$}
\newcommand{\abs}[1]{\left| #1 \right|}
\renewcommand{\dim}[1]{\text{dim}\left( #1 \right)}
\newcommand{\xor}{\oplus}

\setlength{\parindent}{0em}
\setlength{\parskip}{1em}

\begin{document}
	\title{Resolution to Sutner's Conjecture}
	\author{William Boyles\thanks{Department of Mathematics, North Carolina State University, Raleigh, NC 27695 (wmboyle2@ncsu.edu)}}
	\date{\today}
	\maketitle
	
	\section{Introduction}
	Consider a game played on a simple graph $G = (V,E)$ where each vertex consists of a clickable light.
	Clicking any vertex $v$ toggles on on/off state of $v$ and its neighbors.
	One wins the game by finding a sequence of clicks that turns off all the lights.
	When $G$ is a $5 \times 5$ grid, this game was commercially available from Tiger Electronics as \textit{Lights Out}.
	
	Sutner was one of the first to study these games mathematically.
	He showed that for any $G$ the initial configuration of all lights on is solvable \cite{Sutner1989}.
	He also found that when $d(G) = \dim{\ker{(A + I)}}$ over the field $\Z_2$, where $A$ is the adjacency matrix of $G$, is 0 all initial configurations are solvable.
	In particular, 1 out of every $2^{d(G)}$ initial configurations are solvable, while each solvable configuration has $2^{d(G)}$ distinct solutions \cite{Sutner1989}.
	When investigating $n \times n$ grid graphs, Sutner conjectured the following relationship:
	\begin{align*}
		d_{2n+1} &= 2d_n + \delta_n \text{, } \delta_n \in \{0,2\} \\
		\delta_{2n+1} &= \delta_n,
	\end{align*}
	where $d_n = d(G)$ for $G$ an $n \times n$ grid graph \cite{Sutner1989}.
	
	We resolve this conjecture in the affirmative.
	We use results from Sutner that give the nullity of a $n \times n$ board as the GCD of two polynomials in the ring $\Z_2[x]$ \cite{Sutner96sigma-automataand}.
	We then apply identities from Hunziker, Machiavelo, and Park that relate the polynomials $(2n+1) \times (2n+1)$ grids and $n \times n$ grids \cite{HUNZIKER2004465}.
	We then apply a result from Ore about the GCD of two products \cite{ore_number_theory}.
	Together, these results allow us to prove Sutner's conjecture.
	We then go further and show for exactly which values of $n$ $\delta_n$ is 0 or 2.
	
	\section{Fibonacci Polynomials}
	Sutner showed how to calculate $d_n$ as the degree of the GCD of two polynomials in $\Z_2[x]$ \cite{Sutner96sigma-automataand}.
	In this section, we will establish some divisibility properties of these polynomials.
	
	\begin{theorem}[Sutner]\label{Sutner_gcd}
		Let $f_n(x)$ be the polynomial in the ring $\Z_2[x]$ defined recursively by
		\begin{equation*}
			f_n(x) = \begin{cases}
				0 & n=0 \\
				1 & n=1 \\
				xf_{n-1}(x) + f_{n-2}(x) & \text{otherwise }.
			\end{cases}
		\end{equation*}
		Then for all $n \in \N$.
		\begin{equation*}
			d_n = \deg{\gcd\left(f_{n+1}(x), f_{n+1}(x+1)\right)}.
		\end{equation*}
	\end{theorem}
	These polynomials $f$ are often referred to as Fibonacci polynomial because when defined over the reals, evaluating $f_n(x)$ at $x=1$ gives the $n$th Fibonacci number.

	The recursive definition given in Theorem \ref{Sutner_gcd} provides a brute force approach to calculate $f_n(x)$.
	However, Hunziker, Machiavelo, and Park show the following identity that makes calculating $f_n(x)$ easier when $n$ is divisible by powers of 2 \cite{HUNZIKER2004465}.
	
	\begin{theorem}[Hunziker, Machiavelo, and Park]\label{HMP_identity}
		Let $n = b\cdot2^{k}$ where $b$ and $k$ are non-negative integers.
		Then
		\begin{equation*}
			f_n(x) = x^{2^{k}-1} f_{b}^{2^{k}}(x).
		\end{equation*}
	\end{theorem}

	In particular, we will use this result to relate $f_{2n+2}(x)$ and $f_{4n+4}(x)$ to $f_{n+1}(x)$.
	\begin{corollary}\label{cor1}
		The following identities hold:
		\begin{align*}
			f_{2n+2}(x) &= xf_{n+1}^2(x) \\
			f_{4n+4}(x) &= x^3f_{n+1}^4(x).
		\end{align*}
	\end{corollary}
	\begin{proof}
		Notice that $2n+2 = (n+1)2^1$ and $4n+4 = (n+1)2^{2}$.
		Thus, our desired identities follow from Theorem \ref{HMP_identity}.
	\end{proof}

	Now that we have a way to express $f_{2n+2}(x)$ and $f_{4n+4}(x)$ as a product of $f_{n+1}(x)$ and a power of $x$, we simply need a way to express the GCD of products so we can relate $d_{2n+1}$ and $d_n$.
	This is where a number-theoretic result from Ore comes in handy \cite{ore_number_theory}.
	
	\begin{theorem}[Ore]\label{Ore_gcd}
		Let $a$, $b$, $c$, and $d$ be integers.
		Let $(a,b)$ denote $\gcd{(a,b)}$.
		Then
		\begin{equation*}
			(ab,cd) = (a,c)(b,d)\left(\frac{a}{(a,c)},\frac{d}{(b,d)}\right)\left(\frac{c}{(a,c)},\frac{b}{(b,d)}\right).
		\end{equation*}
	\end{theorem}

	Although Ore's result deals specifically with integers, both the integers and $\Z_2[x]$ are Euclidean domains, so the result will still hold.
	
	Hunziker, Machiavelo, and Park also showed the following identity \cite{HUNZIKER2004465}.
	\begin{theorem}[Hunziker, Machiavelo, and Park]\label{HMP_gcd}
		A polynomial $\tau(x)$ in $\Z_2[x]$ divides both $f_n(x)$ and $f_m(x)$ if and only if it divides $f_{\gcd{(m,n)}}$.
		In particular,
		\begin{equation*}
			\gcd{\left(f_m(x), f_n(x)\right)} = f_{\gcd{(m,n)}}(x).
		\end{equation*}
	\end{theorem}

	We specifically will use the following corollary:
	\begin{corollary}\label{cor_HMP_gcd}
		For some polynomial $\tau(x)$ in $\Z_2[x]$, let $n \geq 0$ be the smallest integer such that $\tau(x)$ divides $f_n(x)$.
		Then for all $m \geq 0$, $tau(x)$ divides $f_m(x)$ if and only if $n$ divides $m$.
	\end{corollary}
	\begin{proof}
		Let $\tau(x)$ be some polynomial in $\Z_2[x]$.
		Let $f_n(x)$ be the smallest Fibonacci polynomial such that $\tau(x)$ divides $f_n(x)$.
		
		Assume that $\tau(x)$ divides $f_m(x)$ for some number $m$.
		Then $\tau(x)$ is a common factor of $f_m(x)$ and $f_n(x)$, so Theorem \ref{HMP_gcd} tells us that $\tau(x)$ divides $f_{\gcd{(m,n)}}(x)$.
		Since $f_n(x)$ is the smallest Fibonacci polynomial that is divisible by $\tau(x)$,
		\begin{equation*}
			\gcd{(m,n)} \geq n.
		\end{equation*}
		This inequality only holds if $\gcd{(m,n)} = n$.
		Thus, $m$ must be a multiple of $n$ as desired.
		
		Now assume that $m$ is a multiple of $n$.
		Then $\gcd{(m,n)} = n$.
		Theorem \ref{HMP_gcd} tells us
		\begin{equation*}
			\gcd{\left(f_m(x), f_n(x)\right)} = f_{\gcd{(m,n)}}(x) = f_n(x).
		\end{equation*}
		Since $\tau(x)$ divides $f_n(x)$, and $f_n(x)$ is the GCD of $f_m(x)$ and $f_n(x)$, $\tau(x)$ must also divide $f_m(x)$, as desired.
	\end{proof}

	In particular, we will use the following instances of Corollary \ref{cor_HMP_gcd} to determine when $\delta_n$ is 0 or 2.
	
	\begin{corollary}\label{cor_cor_HMP_gcd}
		The following are true:
		\begin{enumerate}[label=(\roman*)]
			\item $x$ divides $f_n(x)$ if and only if $n \equiv 0 \mod 2$.
			\item $x+1$ divides $f_n(x+1)$ if and only if $n \equiv 0 \mod 2$.
			\item $x+1$ divides $f_n(x)$ if and only if $n \equiv 0 \mod 3$.
			\item $x$ divides $f_n(x+1)$ if and only if $n \equiv 0 \mod 3$.
		\end{enumerate}
	\end{corollary}
	\begin{proof}
		Notice,
		\begin{enumerate}[label=(\roman*)]
			\item
				We find that $f_2(x) = x$ is the smallest Fibonacci polynomial divisible by $x$, so we apply Corollary \ref{cor_HMP_gcd} to get the desired result.
			\item
				Follows from (i) by substituting $x+1$ for $x$.
			\item
				We find that $f_3(x) = (x+1)^2$ is the smallest Fibonacci polynomial divisible by $x+1$, so we apply Corollary \ref{cor_HMP_gcd} to get the desired result.
			\item
				Follows from (iii) by substituting $x+1$ for $x$.
		\end{enumerate}
	\end{proof}

	\section{Proof of Sutner's Conjecture}
	Finally, we are ready to prove Sutner's conjecture \cite{Sutner1989}. 
	\begin{theorem}\label{sutners-thm}
		For all $n \in \N$,
		\begin{equation*}
			d_{2n+1} = 2d_n + \delta_n,
		\end{equation*}
		where $\delta_n \in \{0,2\}$, and $\delta_{2n+1} = \delta_n$.
	\end{theorem}
	\begin{proof}
		Let $(a,b)$ denote $\gcd{(a,b)}$.
		Applying the results from Theorems \ref{Sutner_gcd}, \ref{HMP_identity}, and \ref{Ore_gcd},
		\begin{align*}
			d_{2n+1} &= \deg \left(f_{2n+2}(x), f_{2n+2}(x+1)\right) \\
				&= \deg \left(xf^2_{n+1}(x), (x+1)f^2_{n+1}(x+1)\right) \\
				&= \deg (x,x+1) \left(f^2_{n+1}(x),f^2_{n+1}(x+1)\right) \left(\frac{x+1}{(x,x+1)},\frac{f^2_{n+1}(x)}{(f^2_{n+1}(x),f^2_{n+1}(x+1))}\right) \left(\frac{x}{(x,x+1)},\frac{f^2_{n+1}(x+1)}{(f^2_{n+1}(x),f^2_{n+1}(x+1))}\right) \\
				&= \deg \left(f_{n+1}(x),f_{n+1}(x+1)\right)^2 \left(x+1,\frac{f^2_{n+1}(x)}{(f_{n+1}(x),f_{n+1}(x+1))^2}\right) \left(x,\frac{f^2_{n+1}(x+1)}{(f_{n+1}(x),f_{n+1}(x+1))^2}\right) \\
				&= \deg \left(f_{n+1}(x),f_{n+1}(x+1)\right)^2 \left(x+1,\frac{f_{n+1}(x)}{(f_{n+1}(x),f_{n+1}(x+1))}\right) \left(x,\frac{f_{n+1}(x+1)}{(f_{n+1}(x),f_{n+1}(x+1))}\right) \\
				&= 2d_n + \deg\left(x+1,\frac{f_{n+1}(x)}{(f_{n+1}(x),f_{n+1}(x+1))}\right) \left(x,\frac{f_{n+1}(x+1)}{(f_{n+1}(x),f_{n+1}(x+1))}\right).
		\end{align*}
		Notice that if we substitute $x+1$ for $x$,
		\begin{equation*}
			\left(x+1,\frac{f_{n+1}(x)}{(f_{n+1}(x+1),f_{n+1}(x))}\right) \text{ becomes } \left(x,\frac{f_{n+1}(x+1)}{(f_{n+1}(x),f_{n+1}(x+1))}\right).
		\end{equation*}
		Thus, we see that these two remaining GCD terms in our expression for $d_{2n+1}$ are either both 1 or not 1 simultaneously.
		This means we can further simplify to
		\begin{equation*}
			d_{2n+1} = 2d_{n} + 2\deg \left(x,\frac{f_{n+1}(x+1)}{(f_{n+1}(x),f_{n+1}(x+1))}\right).
		\end{equation*}
		So, we see that
		\begin{equation*}
			d_{2n+1} = 2d_{n} + \delta_n \text{, where }\delta_n = 2\deg \left(x,\frac{f_{n+1}(x+1)}{(f_{n+1}(x),f_{n+1}(x+1))}\right).
		\end{equation*}
		Thus, $\delta_n \in \{0,2\}$ as desired.
		
		What remains is to show that $\delta_{n} = \delta_{2n+1}$.
		Applying Corollary \ref{cor1},
		\begin{align*}
			d_{4n+3} &= \deg \left(x^3f^4_{n+1}(x),(x+1)^3f^4_{n+1}(x+1)\right) \\
				&= \deg \left(x^3,(x+1)^3\right) \left(f^4_{n+1}(x),f^4_{n+1}(x+1)\right) \left(x^3,\frac{f^4_{n+1}(x+1)}{(f^4_{n+1}(x),f^4_{n+1}(x+1))}\right) \left((x+1)^3,\frac{f^4_{n+1}(x)}{(f^4_{n+1}(x),f^4_{n+1}(x+1))}\right) \\
				&= \deg \left(f_{n+1}(x),f_{n+1}(x+1)\right)^4 \left(x^3,\frac{f^4_{n+1}(x+1)}{(f_{n+1}(x),f_{n+1}(x+1))^4}\right) \left((x+1)^3,\frac{f^4_{n+1}(x)}{(f_{n+1}(x),f_{n+1}(x+1))^4}\right) \\
				&= \deg \left(f_{n+1}(x),f_{n+1}(x+1)\right)^4 \left(x^3,\frac{f^3_{n+1}(x+1)}{(f_{n+1}(x),f_{n+1}(x+1))^3}\right) \left((x+1)^3,\frac{f^3_{n+1}(x)}{(f_{n+1}(x),f_{n+1}(x+1))^3}\right) \\
				&= \deg \left(f_{n+1}(x),f_{n+1}(x+1)\right)^4 \left(x,\frac{f_{n+1}(x+1)}{(f_{n+1}(x),f_{n+1}(x+1))}\right)^3 \left((x+1),\frac{f_{n+1}(x)}{(f_{n+1}(x),f_{n+1}(x+1))}\right)^3 \\
				&= 4d_n + 3\delta_n.
		\end{align*}
		Also, from our work previously in this proof,
		\begin{align*}
			d_{4n+3} &= d_{2(2n+1) + 1} \\
				&= 2 d_{2n+1} + \delta_{2n+1} \\
				&= 2 \left(2d_{n} + \delta_{n}\right) + \delta_{2n+1} \\
				&= 4d_{n} + 2\delta_{n} + \delta_{2n+1}.
		\end{align*}
		For these two expressions for $d_{4n+3}$ to be equal, we must have $\delta_{2n+1} = \delta_n$, as desired.
	\end{proof}

	This result seems to have been proven prior by Yamagishi \cite{YAMAGISHI20151}.
	However, Yamagishi does not mention the connection to Sutner's conjecture, and the proof provided is not as direct as the one we provide.
	
	\section{Extended Results}
	Theorem \ref{sutners-thm} proves Sutner's conjecture as stated and even gives a formula for finding $\delta_n$.
	However, this formula is somewhat messy, containing one polynomial division and two polynomial GCDs.
	We can improve this formula to just a modulo operation on $n$.
	We'll do so by using the divisibility properties established in Corollary \ref{cor_cor_HMP_gcd}.
	
	\begin{theorem}\label{when-is-deltan-2}
		The value of $\delta_n$ is 2 if and only if $n+1$ is divisible by 3.
	\end{theorem}
	\begin{proof}
		From our work in Theorem \ref{sutners-thm}, we know that
		\begin{equation*}
			\delta_n = 2\deg\left(x+1,\frac{f_{n+1}(x)}{(f_{n+1}(x),f_{n+1}(x+1))}\right).
		\end{equation*}
		So we see that $\delta_n$ is 2 exactly when $f_{n+1}(x)$ can be divided without remainder by $x+1$ more times than $f_{n+1}(x+1)$.
		
		For $n+1$ is not divisible by 3, Corollary \ref{cor_cor_HMP_gcd} tells us that $f_{n+1}(x)$ is not divisible by $x+1$.
		So in this case, $\delta_n = 0$, as desired.
		
		For $n+1$ divisible by 3, let $n+1 = b \cdot 2^k$ for some integers $b, k \geq 0$ where $b$ is odd.
		Notice that since $n+1$ is divisible by 3, $b$ must also be divisible by 3.
		Applying Corollary \ref{cor1},
		\begin{equation*}
			f_{n+1}(x) = x^{2^k - 1}f_{b}^{2^k}(x) \text{ and } f_{n+1}(x+1) = (x+1)^{2^k - 1}f_{b}^{2^k}(x+1).
		\end{equation*}
		Since $b$ is an odd multiple of 3, Corollary \ref{cor_cor_HMP_gcd} tell us that $x+1$ divides $f_b(x)$, but $x+1$ does not divide $f_b(x+1)$.
		So,
		\begin{equation*}
			f_{n+1}(x) = x^{2^k - 1}(x+1)^{2^k}g^{2^k}(x) \text{ and } f_{n+1}(x+1) = (x+1)^{2^k - 1}x^{2^k}g^{2^k}(x+1),
		\end{equation*} 
		for some $g(x) \in \Z_2[x]$, where $g(x)$ and $g(x+1)$ are both divisible by neither $x$ nor $x+1$.
		So, we see that $f_{n+1}(x)$ can be divided without remainder by $x+1$ one more time than $f_{n+1}(x+1)$.
		So, $\delta_n = 2$, as desired.
	\end{proof}

	\section{Future Work}
	There are many other relationships with $d_n$, some of which are yet to be proven.
	For example, Sutner mentions that for all $k \in \N$, $d_{2^k - 1} = 0$ \cite{Sutner1989}.
	We believe that the following relationships hold, but are unaware of a proof.
	
	\begin{conjecture}\label{conj-all-2}
		There are infinitely many $n$ such that $d_n = 2$.
		In particular, for all $k \in \N$, $d_{2\cdot 3^{k} - 1} = 2$.
	\end{conjecture}
	This conjecture is similar to Sutner's result that shows there are infinitely many $n$ such that $d_n = 0$.
	
	\begin{conjecture}\label{conj-powers}
		Let $a$ be an odd natural number.
		If $a$ is not divisible by 21, then for all $k \in \N$,
		\begin{equation*}
			d_{a^k - 1} = d_{a-1}.
		\end{equation*}
	\end{conjecture}
	Goshima and Yamagishi conjectured a similar statement on tori instead of grids and for $a$ prime \cite{GOSHIMAYAMAGISHI2010}.
	
	\begin{theorem}
		The case of $a=3$ for Conjecture \ref{conj-powers} and \ref{conj-all-2} are equivalent.
	\end{theorem}
	\begin{proof}
		For $a=3$, Conjecture \ref{conj-powers} says that for all $k \in \N$,
		\begin{equation*}
			d_{3^k - 1} = d_{3-1} = 0.
		\end{equation*}
		Since $3^k$ is divisible by 3, Theorem \ref{when-is-deltan-2} tells us that $\delta_{3^k - 1} = 2$.
		So, applying Theorem \ref{sutners-thm},
		\begin{equation*}
			d_{2\cdot3^{k} - 1} = 2d_{3^k - 1} + \delta_{3^k - 1} = 2,
		\end{equation*}
		exactly what Conjecture \ref{conj-all-2} states.
		Apply all the same results in reverse to shows that Conjecture \ref{conj-powers} implies \ref{conj-all-2}.
	\end{proof}
	
	\newpage
	\bibliography{refs.bib}
	\bibliographystyle{amsplain}
\end{document}