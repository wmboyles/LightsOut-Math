\documentclass[a4paper]{article}

\usepackage{amsfonts, amsmath, amssymb, amsthm}
\usepackage{graphicx}
\usepackage{fullpage}
\usepackage{float}

\newtheorem{lemma}{Lemma}
\newtheorem{theorem}{Theorem}
\newtheorem{corollary}{Corollary}
\newtheorem{conjecture}{Conjecture}
\newtheorem{example}{Example}

\newcommand{\Z}{\mathbb{Z}}
\newcommand{\N}{\mathbb{N}}
\renewcommand{\qedsymbol}{$\blacksquare$}
\newcommand{\abs}[1]{\left| #1 \right|}
\renewcommand{\dim}[1]{\text{dim}\left( #1 \right)}
\newcommand{\xor}{\oplus}

\setlength{\parindent}{0em}
\setlength{\parskip}{1em}

\begin{document}
	\title{Resolution to Sutner's Conjecture}
	\author{William Boyles}
	\date{\today}
	\maketitle
	
	\section{Introduction}
	We resolve a conjecture first stated by Sutner in 1989 about the nullity of \textit{Lights Out} boards of size $2n + 1$ in the affirmative \cite{Sutner1989}.
	We use results from Sutner that give the nullity of a $n \times n$ board as the GCD of two polynomials in $\Z_2[x]$ \cite{Sutner96sigma-automataand}.
	We then apply identities from Hunziker, Machiavelo, and Park that relate these polynomials for $(2n+1) \times (2n+1)$ boards to those for $n \times n$ boards \cite{HUNZIKER2004465}.
	Finally, we use a result from Ore about the GCD of products \cite{ore_number_theory}.
	Together, these results allow us to prove Sutner's conjecture and provide conditions under which each case occurs.
	
	\section{Preliminary Results}
	Let $d(n)$ be the nullity of an $n \times n$ \textit{Light Out board}.
	Sutner showed how to calculate $d(n)$ as the degree of the GCD of two polynomials \cite{Sutner96sigma-automataand}.
	
	\begin{theorem}[Sutner]\label{Sutner_gcd}
		Let $f_n(x)$ be the degree $n$ polynomial in the ring $\Z_2[x]$ defined recursively by
		\begin{equation*}
			f_n(x) = \begin{cases}
				1 & n=0 \\
				x & n=1 \\
				xf_{n-1}(x) + f_{n-2}(x) & \text{otherwise }.
			\end{cases}
		\end{equation*}
		Then for all $n \in \N$.
		\begin{equation*}
			d(n) = \deg{\gcd\left(f_{n}(x), f_{n}(x+1)\right)}.
		\end{equation*}
	\end{theorem}

	This recursive definition gives a brute force approach to calculate $f_n(x)$.
	However, Hunziker, Machiavelo, and Park show the following identity that can make calculating certain $f_n(x)$ easier.
	
	\begin{theorem}[Hunziker, Machiavelo, and Park]\label{HMP_identity}
		Let $n = b\cdot2^{k-1} - 1$ where $b, k \in \N$.
		Then
		\begin{equation*}
			f_n(x) = x^{2^{k-1}-1} f_{b-1}^{2^{k-1}}(x).
		\end{equation*}
	\end{theorem}

	In particular, we will use this result to relate $f_{2n+1}(x)$ and $f_{4n+3}(x)$ to $f_n(x)$.
	\begin{corollary}\label{cor1}
		The following identities hold
		\begin{align*}
			f_{2n+1}(x) &= xf_{n}^2(x) \\
			f_{4n+3}(x) &= x^3f_{n}^4(x).
		\end{align*}
	\end{corollary}
	\begin{proof}
		Notice that $2n+1 = (n+1)2^{2-1} - 1$ and $4n+3 = (n+1)2^{3-1} - 1$.
		Thus, our desired identities follow from Theorem \ref{HMP_identity}.
	\end{proof}

	Now that we have a way to express $f_{2n+1}(x)$ and $f_{4n+3}(x)$ in terms of the product of $f_{n}(x)$ terms and $x$, we simply need a way to express the GCD of products.
	This is where a number-theoretic result from Ore comes in handy \cite{ore_number_theory}.
	
	\begin{theorem}[Ore]\label{Ore_gcd}
		Let $(a,b)$ denote $\gcd{(a,b)}$.
		Then
		\begin{equation*}
			(ab,cd) = (a,c)(b,d)\left(\frac{a}{(a,c)},\frac{d}{(b,d)}\right)\left(\frac{c}{(a,c)},\frac{b}{(b,d)}\right).
		\end{equation*}
	\end{theorem}

	\section{Sutner's Conjecture}
	Finally, we are ready to state and prove Sutner's conjecture \cite{Sutner1989}. 
	\begin{theorem}
		For all $n \in \N$,
		\begin{equation*}
			d(2n+1) = 2d(n) + \delta_n,
		\end{equation*}
		and $\delta_{2n+1} = \delta_n$.
	\end{theorem}
	\begin{proof}
		Applying the results from Theorems \ref{Sutner_gcd}, \ref{HMP_identity}, and \ref{Ore_gcd},
		\begin{align*}
			d(2n+1) &= \deg \left(f_{2n+1}(x), f_{2n+1}(x+1)\right) \\
				&= \deg \left(xf^2_n(x), (x+1)f^2_n(x+1)\right) \\
				&= \deg (x,x+1)\left(f^2_n(x),f^2_n(x+1)\right)\left(\frac{x+1}{(x,x+1)},\frac{f^2_n(x)}{(f^2_n(x),f^2_n(x+1))}\right)\left(\frac{x}{(x,x+1)},\frac{f^2_n(x+1)}{(f^2_n(x),f^2_n(x+1))}\right) \\
				&= \deg \left(f_n(x),f_n(x+1)\right)^2\left(x+1,\frac{f^2_n(x)}{(f_n(x),f_n(x+1))^2}\right)\left(x,\frac{f^2_n(x+1)}{(f_n(x),f_n(x+1))^2}\right) \\
				&= \deg \left(f_n(x),f_n(x+1)\right)^2\left(x+1,\frac{f_n(x)}{(f_n(x),f_n(x+1))}\right)\left(x,\frac{f_n(x+1)}{(f_n(x),f_n(x+1))}\right) \\
				&= 2d(n) + \deg\left(x+1,\frac{f_n(x)}{(f_n(x),f_n(x+1))}\right)\left(x,\frac{f_n(x+1)}{(f_n(x),f_n(x+1))}\right).
		\end{align*}
		Notice that if we substitute $x+1$ for $x$, 
		\begin{equation*}
			\left(x+1,\frac{f_n(x)}{(f_n(x+1),f_n(x)}\right) \text{ becomes } \left(x,\frac{f_n(x+1)}{(f_n(x),f_n(x+1))}\right).
		\end{equation*}
		Thus, we see that these two remaining GCD terms are either both 1 nor not 1 simultaneously.
		This means we can further simplify to
		\begin{equation*}
			d(2n+1) = 2d(n) + 2\deg\left(x,\frac{f_n(x+1)}{(f_n(x),f_n(x+1))}\right).
		\end{equation*}
		So, we see that
		\begin{equation*}
			d(2n+1) = 2d(n) + \delta_n \text{, where }\delta_n = 2\deg\left(x,\frac{f_n(x+1)}{(f_n(x),f_n(x+1))}\right).
		\end{equation*}
		Thus, $\delta_n \in \{0,2\}$.
		
		Next, we'll calculate $\delta_{2n+1}$.
		Applying Corollary \ref{cor1},
		\begin{align*}
			d(4n+3) &= \deg\left(x^3f^4_n(x), (x+1)^3f^4_n(x+1)\right) \\
				&= \deg\left(x,(x+1)^3\right)\left(f^4_n(x),f^4_n(x+1)\right)\left(x^3,\frac{f^4_n(x+1)}{(f^4_n(x),f^4_n(x+1))}\right)\left((x+1)^3,\frac{f^4_n(x)}{(f^4_n(x),f^4_n(x+1))}\right) \\
				&= \deg\left(f_n(x),f_n(x+1)\right)^4\left(x^3,\frac{f^4_n(x+1)}{(f_n(x),f_n(x+1))^4}\right)\left((x+1)^3,\frac{f^4_n(x)}{(f_n(x),f_n(x+1))^4}\right) \\
				&= \deg\left(f_n(x),f_n(x+1)\right)^4\left(x^3,\frac{f^3_n(x+1)}{(f_n(x),f_n(x+1))^3}\right)\left((x+1)^3,\frac{f^3_n(x)}{(f_n(x),f_n(x+1))^3}\right) \\
				&= \deg\left(f_n(x),f_n(x+1)\right)^4\left(x,\frac{f_n(x+1)}{(f_n(x),f_n(x+1))}\right)^3\left((x+1),\frac{f_n(x)}{(f_n(x),f_n(x+1))}\right)^3 \\
				&= 4d(n) + 3\delta_n.
		\end{align*}
		Also,
		\begin{align*}
			d(4n+3) &= 2d(2n+1) + \delta_{2n+1} \\
				&= 2\left(2d(n) + \delta_n\right) + \delta_{2n+1} \\
				&= 4d(n) + 2\delta_n + \delta_{2n+1}.
		\end{align*}
		For these two expressions for $d(2(2n+1)+1)$ to be equal, we must have $\delta_{2n+1} = \delta_n$.
	\end{proof}
	
	\newpage
	\bibliography{refs.bib}
	\bibliographystyle{amsplain}
\end{document}